% -------------------------------------------------------------------------
% Setup
% -------------------------------------------------------------------------
\documentclass[11pt, aspectratio=149]{beamer}
% Options for aspectratio: 1610, 149, 54, 43 and 32, 169
\usepackage[utf8]{inputenc}
\usepackage[english]{babel}% Alternative: 'norsk'
\usepackage[expansion=false]{microtype}% Fixes to make typography better
\usecolortheme{beaver} % Decent options: beaver, rose, crane
\usepackage{listings}% To include source-code
\usepackage{booktabs}% Professional tables
\usefonttheme{serif}
\usepackage{mathptmx}
\usepackage[scaled=0.9]{helvet}
\usepackage{courier}

\title{Analyzing data with the Pareto set}
\subtitle{A graphical tutorial with examples}
\date{\today}
\author{tommyod @ GitHub}

% -------------------------------------------------------------------------
% Package imports
% -------------------------------------------------------------------------
\usepackage{etoolbox}
\usepackage{graphicx}
\usepackage{tikz}
\usepackage{amsmath}
\usepackage{amsthm}
\usepackage{amsfonts}
\usepackage{amssymb}
\usepackage{mathtools}
\usepackage{graphicx}
\usepackage{hyperref}
\usepackage{listings}
\usepackage[sharp]{easylist}
\usepackage{multicol}
\usepackage{minted}


\usefonttheme{professionalfonts}
\usepackage{fontspec}
\setmainfont{Open Sans}
\setsansfont{Open Sans}
\setmonofont{Ubuntu Mono}
\usefonttheme{serif}

%gets rid of bottom navigation bars
\setbeamertemplate{footline}[frame number]{}

%gets rid of bottom navigation symbols
\setbeamertemplate{navigation symbols}{}

% Set up colors to be used
\definecolor{purered}{RGB}{31,119,180}
\definecolor{titlered}{RGB}{31,119,180}
\definecolor{bggray}{RGB}{242,242,242}
\definecolor{bggraydark}{RGB}{217,217,217}

% Change the default colors

\setbeamercolor*{title}{bg=bggray,fg=titlered}
\AtBeginEnvironment{theorem}{%
	\setbeamercolor{block title}{fg=titlered, bg=bggraydark}
	\setbeamercolor{block body}{fg=black,bg=bggray}
}
\AtBeginEnvironment{proof}{%
	\setbeamercolor{block title}{bg=bggraydark}
	\setbeamercolor{block body}{fg=black,bg=bggray}
}
\AtBeginEnvironment{example}{%
	\setbeamercolor{block title example}{bg=bggraydark}
	\setbeamercolor{block body example}{fg=black,bg=bggray}
}
\AtBeginEnvironment{definition}{%
	\setbeamercolor{block title}{bg=bggraydark}
	\setbeamercolor{block body}{fg=black,bg=bggray}
}

\setbeamercolor{block title example}{bg=bggraydark}
\setbeamercolor{block body example}{fg=black,bg=bggray}
\setbeamercolor{block title}{bg=bggraydark}
\setbeamercolor{block body}{fg=black,bg=bggray}

\setbeamercolor{frametitle}{fg=titlered,bg=bggray}
\setbeamercolor{section in head/foot}{bg=black}
\setbeamercolor{author in head/foot}{bg=black}
\setbeamercolor{date in head/foot}{fg=titlered}


% Spacing for lsits
\newcommand{\listSpace}{0.4em}

% Theorems, equations, definitions setup
\theoremstyle{plain}

\usepackage{etoolbox}
\usepackage{lipsum}

\makeatletter
\patchcmd{\beamer@sectionintoc}
{\vfill}
{\vskip\itemsep}
{}
{}
\makeatother  

\AtBeginSection[]{
	\begin{frame}
		\vfill
		\centering
		\begin{beamercolorbox}[sep=8pt,center,shadow=false,rounded=false]{title}
			\usebeamerfont{title}\insertsectionhead\par%
		\end{beamercolorbox}
		\vfill
	\end{frame}
}

% -------------------------------------------------------------------------
% Document start
% -------------------------------------------------------------------------
\begin{document}
\maketitle

% -------------------------------------------------------------------------

\section{Motivating example: renting an apartment}

% -------------------------------------------------------------------------
\begin{frame}[fragile, t]{The data set of apartments}
	\begin{figure}
		\centering
		\includegraphics[width=0.8\linewidth]{apartments1.pdf}
	\end{figure}
	Consider a data set of $30$ apartments. Which apartments are \emph{best}?
	\vfill
\end{frame}

% -------------------------------------------------------------------------
\begin{frame}[fragile, t]{General directions}
	\begin{figure}
		\centering
		\includegraphics[width=0.8\linewidth]{apartments2.pdf}
	\end{figure}
	Apartments with a high price and low square meters are obviously bad.
	Informally: we want to stay away from the upper-left corner.
	\vfill
\end{frame}

% -------------------------------------------------------------------------
\begin{frame}[fragile, t]{Scalarization}
	\begin{figure}
		\centering
		\includegraphics[width=0.8\linewidth]{apartments3.pdf}
	\end{figure}
	We can choose $\lambda \in [0, 1]$ and minimize a weighted sum of features:
	\begin{equation*}
		f(\text{price}, \text{sqm}) = \lambda \text{price} + (1 -\lambda) (-\text{sqm})
	\end{equation*}
	\vfill
\end{frame}

% -------------------------------------------------------------------------
\begin{frame}[fragile, t]{Scalarization}
	\begin{figure}
		\centering
		\includegraphics[width=0.8\linewidth]{apartments4.pdf}
	\end{figure}
	We can choose $\lambda \in [0, 1]$ and minimize a weighted sum of features:
	\begin{equation*}
	f(\text{price}, \text{sqm}) = \lambda \text{price} + (1 -\lambda) (-\text{sqm})
	\end{equation*}
	\vfill
\end{frame}

% -------------------------------------------------------------------------
\begin{frame}[fragile, t]{Disadvantages of scalarization}
	\begin{figure}
		\centering
		\includegraphics[width=0.8\linewidth]{apartments5.pdf}
	\end{figure}
	The parameter $\lambda$ must be chosen a priori.
	Some interesting apartments cannot be found.
	In higher dimensions:
	\begin{equation*}
	\text{minimize} \quad f(\mathbf{x}) = \sum_i \lambda_i x_i, \quad \text{subject to} \quad \sum_i \lambda_i = 1.
	\end{equation*}
	\vfill
\end{frame}

% -------------------------------------------------------------------------
\begin{frame}[fragile, t]{A better approach: Pareto domination}
	\begin{figure}
		\centering
		\includegraphics[width=0.8\linewidth]{apartments6.pdf}
	\end{figure}
	Consider apartment $\mathbf{x}$ shown in black.
\end{frame}

% -------------------------------------------------------------------------
\begin{frame}[fragile, t]{A better approach: Pareto domination}
	\begin{figure}
		\centering
		\includegraphics[width=0.8\linewidth]{apartments7.pdf}
	\end{figure}
	The shaded region is \emph{dominated} by $\mathbf{x}$, since they have higher price and fewer square meters.
	Apartment $\mathbf{x}$ is unambiguously better. 
\end{frame}

% -------------------------------------------------------------------------
\begin{frame}[fragile, t]{A better approach: Pareto domination}
	\begin{figure}
		\centering
		\includegraphics[width=0.8\linewidth]{apartments8.pdf}
	\end{figure}
	The dominated apartments can safely be moved.
	If $\mathbf{x}$ dominates $\mathbf{y}$, there is no reason to prefer $\mathbf{y}$ over $\mathbf{x}$.
\end{frame}

% -------------------------------------------------------------------------
\begin{frame}[fragile, t]{A better approach: Pareto domination}
	\begin{figure}
		\centering
		\includegraphics[width=0.8\linewidth]{apartments9.pdf}
	\end{figure}
	The size of the data set is reduced from $30$ to $18$ apartments.
\end{frame}

% -------------------------------------------------------------------------
\begin{frame}[fragile, t]{The Pareto set}
	\begin{figure}
		\centering
		\includegraphics[width=0.8\linewidth]{apartments10.pdf}
	\end{figure}
	The \emph{Pareto set} consists of apartments not dominated by any other apartment.
\end{frame}

% -------------------------------------------------------------------------
\begin{frame}[fragile, t]{The Pareto set}
	\begin{figure}
		\centering
		\includegraphics[width=0.8\linewidth]{apartments10.pdf}
	\end{figure}
	The \emph{Pareto set} is also called the ``Pareto frontier'', ``skyline'' or the ``efficient solutions.''
\end{frame}

\section{Definitions}

% -------------------------------------------------------------------------
\begin{frame}[fragile, t]{Domination}
				\begin{figure}
					\centering
					\includegraphics[width=0.8\linewidth]{domination.pdf}
				\end{figure}
	We say that $\mathbf{x}$ \emph{dominates} $\mathbf{y}$ if:
	\begin{easylist}
		# In all dimensions $i$, $\mathbf{x}$ is \emph{at least as good} as $\mathbf{y}$.
		# In at least one dimension $i$, $\mathbf{x}$ is \emph{better} than $\mathbf{y}$.
	\end{easylist}
\end{frame}

% -------------------------------------------------------------------------
\begin{frame}[fragile, t]{Domination}
	\begin{figure}
		\centering
		\includegraphics[width=0.8\linewidth]{domination.pdf}
	\end{figure}
	If $\mathbf{x}$ dominates $\mathbf{y}$ in a minimization problem, we write $\mathbf{x} \prec \mathbf{y}$.
	
	\textbf{Example.} Clearly $(1, 3) \prec (2, 5)$. The points $(1, 3)$ and $(3, 1)$ are \emph{incomparable}, since $(1, 3) \nsucc (3, 1)$ and $(1, 3) \nprec (3, 1)$.
\end{frame}


\section{Example: Buying a new computer}

% -------------------------------------------------------------------------
\begin{frame}[fragile, t]{Computers data set}
	\footnotesize 
\begin{tabular}{lrrrrr}
	\toprule
	name &  screen &  RAM &  HDD &  weight &  price \\
	\midrule
	Apple MacBook Air 13,3 128GB &    13.3 &    8 &  128 &     NaN &   9990 \\
	Asus ZenBook Pure UX430UN-PURE2 &    14.0 &    8 &  256 &   1.300 &   7999 \\
	HP Pavilion Gaming 15-cx0015no &    15.6 &    8 &  256 &   2.220 &   5999 \\
	Huawei D15 (53010TTV) &    14.0 &    8 &  256 &   1.530 &   5495 \\
	Apple MacBook Air 13.3 256GB &    13.3 &    8 &  256 &   1.290 &  12495 \\
	Asus Chromebook C523 &    15.6 &    4 &   32 &   1.430 &   3495 \\
	Huawei MateBook 13 (18140) &    13.0 &    8 &  256 &     NaN &   8995 \\
	Asus ZenBook UX433FN-A6094T &    14.0 &    8 &  256 &   1.300 &   7999 \\
	Microsoft Surface Laptop 2 &    13.5 &    8 &  128 &   1.283 &   7999 \\
	Lenovo Ideapad S145 (81W80028MX) &    15.6 &    8 &  256 &   1.850 &   4690 \\
	Huawei MateBook 13 (51204) &    13.0 &    8 &  512 &   1.300 &   9995 \\
	Apple MacBook Air (Mid 2017) &    13.3 &    8 &  128 &   1.350 &   9199 \\
	Acer Nitro 5 (NH.Q5XED.018) &    15.6 &   16 &  512 &   2.200 &   8499 \\
	\bottomrule
\end{tabular}
\normalsize

\vspace*{1em}
A data set with 13 computers from April 2020.
\end{frame}


\begin{frame}[fragile, t]{Pareto set of computers: RAM, HDD and price}
Computed using the \texttt{paretoset} package in Python:
\vspace*{1em}
\begin{verbatim}
from paretoset import paretoset
mask = paretoset(df_computers[["RAM", "HDD", "price"]], 
                 sense=[max, max, min])
df_computers[mask] # Reduced to 3 / 13 computers
\end{verbatim}
\vspace*{1em}

\footnotesize 
\begin{tabular}{lrrrrr}
	\toprule
	name &  screen &  RAM &  HDD &  weight &  price \\
	\midrule
	Asus Chromebook C523 &    15.6 &    4 &   32 &    1.43 &   3495 \\
	Lenovo Ideapad S145 (81W80028MX) &    15.6 &    8 &  256 &    1.85 &   4690 \\
	Acer Nitro 5 (NH.Q5XED.018) &    15.6 &   16 &  512 &    2.20 &   8499 \\
	\bottomrule
\end{tabular}
\normalsize
\end{frame}


\begin{frame}[fragile, t]{Pareto set of computers: RAM, HDD, weight and price}
	\begin{verbatim}	
mask = paretoset(df_computers[["RAM", "HDD", "weight", "price"]], 
                 sense=[max, max, min, min])
	
df_computers[mask] # Reduced to 9 / 13 computers
\end{verbatim}
	\vspace*{1em}
	
	\footnotesize 
\begin{tabular}{lrrrrr}
	\toprule
	name &  screen &  RAM &  HDD &  weight &  price \\
	\midrule
	Asus ZenBook Pure UX430UN-PURE2 &    14.0 &    8 &  256 &   1.300 &   7999 \\
	Huawei D15 (53010TTV) &    14.0 &    8 &  256 &   1.530 &   5495 \\
	Asus Chromebook C523 &    15.6 &    4 &   32 &   1.430 &   3495 \\
	Huawei MateBook 13 (18140) &    13.0 &    8 &  256 &     \textbf{NaN} &   8995 \\
	Asus ZenBook UX433FN-A6094T &    14.0 &    8 &  256 &   1.300 &   7999 \\
	Microsoft Surface Laptop 2 &    13.5 &    8 &  128 &   1.283 &   7999 \\
	Lenovo Ideapad S145 (81W80028MX) &    15.6 &    8 &  256 &   1.850 &   4690 \\
	Huawei MateBook 13 (51204) &    13.0 &    8 &  512 &   1.300 &   9995 \\
	Acer Nitro 5 (NH.Q5XED.018) &    15.6 &   16 &  512 &   2.200 &   8499 \\
	\bottomrule
\end{tabular}


	\normalsize
\end{frame}



\section{The \texttt{paretoset} Python package}
% -------------------------------------------------------------------------
\begin{frame}[fragile, t]{\texttt{paretoset}}
	If you're interested in Pareto set computations in Python, I've written a library.
	\vspace{1em}
	\begin{easylist}[itemize]
		\ListProperties(Space=\listSpace, Space*=\listSpace)
		# GitHub: \url{https://github.com/tommyod/paretoset}
		# PyPI: \url{https://pypi.org/project/paretoset}
	\end{easylist}
	\vfill
	
	\textbf{Features:}
	\vspace{1em}
	\begin{easylist}[itemize]
		\ListProperties(Space=\listSpace, Space*=\listSpace)
		# Support for pandas DataFrames and NumPy ndarrays.
		# Support for MIN, MAX and DIFFERENT over dimensions.
		# Return distinct or non-distinct data.
		# Fast: handles a million rows in less than a second.
		# Functions related to Pareto sets, e.g. Pareto ranks.
	\end{easylist}
	\vfill
\end{frame}


% -------------------------------------------------------------------------
%\begin{frame}[fragile, t]{References}
%	References for further reading.
%	\vspace{1em}
%	\begin{easylist}[itemize]
%		\ListProperties(Space=\listSpace, Space*=\listSpace)
%		# Wikipedia article on Pareto efficiency.
%		# Boyd et al. \emph{Convex Optimization}
%		# Boyd et al. \emph{Convex Optimization}
%		# Börzsönyi et al. \emph{The Skyline Operator}
%	\end{easylist}
%\end{frame}

\end{document}
